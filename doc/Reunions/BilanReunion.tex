\documentclass{article}
\usepackage[utf8]{inputenc}

\begin{document}
\title{Bilan des réunions}
\author{
    Soufien Benramdhane\\
    Alexandre Da-silva\\
    Romain Tholimet\\
    Johannick Renaud\\
    Antoine Piellard\\
    Mathieu Berbegal
}
\maketitle
\newpage
\section{Première Réunion}
Pendant cette réunion, on a d'abord découpé le projet en plusieurs phases afin
de s'organiser un minimum. Ensuite, on a essayé de voir ce qu'un client et
serveur R-Type pourraient contenir.
\subsection{Les phases du projet}
\subsubsection{Phase 1 (Durée : 1 Semaine)}
\begin{itemize}
  \item{Protocole binaire et conception. À finir avant le suivi !}
\end{itemize}
\subsubsection{Phase 2 (Durée : 1 Semaine)}
\begin{itemize}
  \item{Affichage statique : On voit les les décors et les monstres fixes sans
  pattern précis,}
  \item{Connexion des joueurs avec gestion des rooms (côté serveur),}
  \item{Gestion des événements clavier.}
\end{itemize}
\subsubsection{Phase 3 (Durée : 2 Semaines)}
\begin{itemize}
  \item{Génération des monstres aléatoire,}
  \item{Gestion des patterns,}
  \item{Gestion et scrolling des starfields,}
  \item{Gestion des collisions,}
  \item{Gestion des maps.}
\end{itemize}
\subsubsection{Phase 4 (Durée : 1 Semaine)}
\begin{itemize}
  \item{Phase de debug et de test,}
  \item{Commencement et/ou finition de la GUI,}
  \item{Créer les levels comme dans R-Type,}
  \item{Faire tout comme dans R-Type.}
\end{itemize}
\subsubsection{Apparté}
Le but c'est de respecter au maximum ces phases. On va bien sûr parfois aller
plus vite, parfois moins.
\newpage
\subsection{Vision globale}
\begin{tabular}{| c | c |}
  \hline
  Client & Serveur \\ \hline
  SFML & Quadtree \\ \hline
  Core (Boucle de jeu) & Core \\ \hline
  Abstraction Réseau TCP et UDP & Abstraction Réseau TCP et UDP \\ \hline
  Gestion des animations & Abstraction Thread + Gestion du debug \\ \hline
  GUI (Menu) & Gestion événement \\ \hline
  Gestion des logs & Gestion des rooms \\ \hline
  Abstraction Son, Image, Thread & TChat \\ \hline
  Banque Multimédia + Factory & Fichiers de données \\ \hline
  & Gestion des joueurs (Bonus, vie, etc...) \\ \hline
  & Gestion du temps, vitesse \\ \hline
\end{tabular}
\subsection{Les questions qu'on se posait}
\begin{itemize}
  \item{Est-ce que l'on envoit à chaque fois toutes les positions ? (Du coup, on
      a décidé que non)}
  \item{Comprendre le fonctionnement du quadtree (on a décidé qu'on en faisait
      pas.)}
  \item{Doit-on mettre un système de room?(On sait qu'il en faut un)}
\end{itemize}
\newpage
\section{Deuxième Réunion}
Pendant cette réunion, on a définit les bases du protocoles et établit quelques
énumrations.
\subsection{Formattage d'un paquet}
\begin{tabular}{| c | c | c | c | c | c |}
  \hline
  Nbr de bit & 8 & 2 & 2 & 4 & 253 octets \\ \hline
  Parties & Size & Type & TT (transaction type) & Sous-type & Data \\ \hline
\end{tabular}
\subsection{Le TT}
\begin{itemize}
  \item{REQUEST}
  \item{ERROR}
  \item{REPLY}
  \item{ACK}
\end{itemize}
\subsection{Les types}
\subsubsection{Connection 0x0}
\begin{itemize}
  \item{AUTH}
  \item{DISC}
  \item{ALIVE}
\end{itemize}
\subsubsection{Room 0x1}
\begin{itemize}
  \item{PLAYER}
  \item{JOIN}
  \item{LEAVE}
  \item{LIST}
\end{itemize}
\subsubsection{Game 0x2}
\begin{itemize}
  \item{PLAYER}
  \item{MAP}
  \item{STARFIELD}
  \item{ENEMY}
\end{itemize}
\subsection{Apparté}
Pendant cette réunion, on a décidé :
\begin{itemize}
  \item{Que pour la gestion des collisions on
  abandonne les quad trees et qu'on découpe la
    map sans faire d'arbre,}
  \item{Qu'on envoit la map à chaque début de
  level,}
  \item{Que toutes les rooms sont crées à
  l'avance,}
  \item{Qu'on peut voir le nombre de joueurs
  dans les rooms mais pas leur login etc...
  Pour les voir, il faut cliquer ou passer
  dessus,}
  \item{Que dans le champ Data pour l'envoi
  d'une map on mette 1 char pour le numéro
  du paquet, 1 char pour la ligne
  correspondante et le reste remplit des
  données.}
\end{itemize}
\section{Norme}
Voici les règles de normes établies avec le groupe :
\subsection{Dans le code}
\begin{itemize}
\item{Pas de header emacs.}
\item{Les méthodes : getMembers.}
\item{Les interfaces : IClass / Les abstraites : AClass.}
\item{Les variables membres : nomDeVariable\_}
\item{Extension des fichiers headers : .hpp}
\item{Les variables container : nameList, roomVector ou utilisation de typedef}
\end{itemize}
\subsection{Dans les commits}
\begin{verbatim}
Message ::= '[' Part ']' Verb msg '.'
Part	   ::= ['A' - 'Z']['a' - 'z' | '0' - '9' | '_']
Verb	   ::= <<verb in past (je sais c'est pas de la BNF mais fuck! =3)>>
msg	   ::= classic message
\end{verbatim}
\end{document}
